% Load environment
\documentclass[12pt]{article}
%\usepackage[cam]{crop}
\usepackage{tikz,tkz-euclide,tkz-base}
\usetkzobj{all}
\usetikzlibrary{decorations.pathmorphing}
\usepackage[portuges]{babel}
\usepackage{amssymb,euler,fontspec,multicol,xcolor,pstricks,setspace,amsmath,enumerate,pstricks,ifthen,multido,titlesec,pst-eucl,pgfplots}
\usepackage{xunicode}
\defaultfontfeatures{Mapping=tex-text}
\setmainfont[Path = Base/fonts/, Scale=1]{Roboto-Light.ttf}
\setsansfont[Path = Base/fonts/, Scale=1]{Roboto-Bold.ttf}
\setmonofont[Path = Base/fonts/, Scale=1]{Roboto-Thin.ttf}
\XeTeXinputencoding  Latin1
\usepackage[paperwidth=21cm,paperheight=29.7cm,margin=1cm,bottom=1.5cm]{geometry}
\onehalfspacing
\usepackage{amsmath,amsthm,amsfonts,amssymb,amscd}


% %Ambiente Exemplo
% \newenvironment{exe}{\addtocounter{problema}{1}%
% \vskip\baselineskip%
% \noindent
% \begin{minipage}{\columnwidth}
% \noindent\begin{tikzpicture}[remember picture,baseline=(current bounding box.south)]\node [draw,rectangle, rounded corners,right,fill,text height=10pt,font=\small] (ex) at (0,0) {\color{white}\scalebox{1.5}{\textsf{E}}\textsf{xemplo \theproblema}};\end{tikzpicture}\hspace{1em}}{\end{minipage}\vskip.25\baselineskip%
% \noindent\tikz \draw [dashed] (0,0) -- (19,0);}

%Ambiente Respostas
\newenvironment{res}{
    \vskip\baselineskip%
    \noindent
    \begin{minipage}{\columnwidth}
    \noindent\begin{tikzpicture}[remember picture,baseline=(current bounding box.south)]\node [draw,rectangle, rounded corners,right,fill,text height=10pt,font=\small] (ex) at (0,0) {\color{white}\scalebox{1.0}{\textsf{R}}\textsf{espostas}};\end{tikzpicture}\hspace{1em}}{\end{minipage}\vskip.25\baselineskip%
    % \noindent\tikz \draw [dashed] (0,0) -- (19,0);
}

%Ambiente Problema
\newenvironment{prob}{\addtocounter{problema}{1}%
\vskip\baselineskip%
\noindent
\begin{minipage}{\columnwidth}
    \noindent
    \begin{tikzpicture}[remember picture,baseline=(current bounding box.south)]
        \node [draw,rectangle, rounded corners,right,text height=10pt,font=\small]
        (ex) at (0,0)
        {\scalebox{1.0}{\textsf{P}}\textsf{roblema \theproblema}};
    \end{tikzpicture}\hspace{1em}
\end{minipage}
}


%Ambiente Solu\c{c}\~{a}o
\newenvironment{solu}{%
\vskip.25\baselineskip\noindent%
\scalebox{1.0}{\textsf{S}}\textsf{{olu\c{c}\~{a}o:}}\hspace{1em}}{\vskip.25\baselineskip\hfill$\square$\vskip.25\baselineskip}



\theoremstyle{plain}
\newtheorem{teo}{Teorema}[section]
\newtheorem{lema}[teo]{Lema}
\newtheorem{propo}[teo]{Proposi\c{c}\~{a}o}
\newtheorem{coro}[teo]{Corol\'{a}rio}
%\newtheorem{prob}{Problema}[section]
\newtheorem{prop}{Exercício}[section]

%\theoremstyle{defin}

\newtheorem{defin}[teo]{Defini\c{c}\~{a}o}
\newtheorem{obser}[teo]{Observa\c{c}\~{a}o}
\newtheorem{exem}[teo]{Exemplo}
\newcommand{\field}[1]{\mathbb{#1}}
\newcommand{\NN}{\field{N}}
\newcommand{\ZZ}{\field{Z}}
\newcommand{\QQ}{\field{Q}}
\newcommand{\RR}{\field{R}}
\newcommand{\CC}{\field{C}}



\def\sen{\mathop{\rm sen}\nolimits}
\def\Cal#1{{\cal #1}}
\def\fim{\hfill\hbox{$\square$}}
\def\demo{{\it Demonstra\c{c}\~{a}o:\ }}



\setlength{\parindent}{2em}
\newcommand{\paginabranco}{\newpage{\thispagestyle{empty}\cleardoublepage}}
\newcommand{\cabeca}[4]{%
\noindent\begin{tikzpicture}[scale=1]
\node [left,rotate=0] at (19,1.5){\XeTeXpicfile Base/logo.png width 3cm
};
\node [right,text width=14cm] at (0,1.5) {{\text{\scalebox{1.2}{\textsf{#1}}}}};
\node [right] at (0,1) {\scriptsize Instrutor #2 };
\node [above] at (9.5,-2) {\scalebox{2}{\sf#3}};
\draw [->] (0,0) -- (19,0);
\node [right] at (0,.5) {\scriptsize #4};
\end{tikzpicture}
}

\titleformat{\section}[block] {\normalfont\sffamily} {\thesection}{.5em}{\titlerule\\[.8ex]\sffamily}


% Theorem-like environments
\newtheorem{theorem}{Teorema}
\newtheorem{proposition}[theorem]{Proposi\c{c}\~{a}o}
\newtheorem{lemma}[theorem]{Lema}
\newtheorem{definition}[theorem]{Defini\c{c}\~{a}o}
\newtheorem{corollary}[theorem]{Corol\'{a}rio}
\newtheorem{example}[theorem]{Exemplo}

\renewcommand{\ge}{\geqslant}
\renewcommand{\geq}{\geqslant}

\renewcommand{\le}{\leqslant}
\renewcommand{\leq}{\leqslant}


% Begin document
\begin{document}\pagestyle{empty}
% {Nome da Aula}{Professores}{Matéria da Aula}{data}
\cabeca{Aula 1}{Maronato}{Funções}{21/12/2016}
\newcounter{problema}


\begin{prob}
Sendo $f\left( x\right) = x^{2}-x$ e $g\left( x\right) = x+3$, calcule:
\begin{multicols}{2}
\begin{enumerate}[(a)]
\item    $f\left( g\left( x\right) \right) $
\item    $g\left( f\left( x\right) \right) $
\end{enumerate}
\end{multicols}
\end{prob}

\begin{prob}
Sendo $f\left( x\right) = x^{2}$, calcule $\dfrac{f\left( x+h\right) -f\left( x\right) }{h}$
\end{prob}

\begin{prob}
Dado o gr\'{a}fico de $f\left( x\right)$, calcule $f\left( 20\right)$


\begin{tikzpicture}
        \begin{axis}[ xmin=-4,xmax=1, xlabel=$x$, ylabel=$f\left( x\right)$,grid=both, mark=none ]
    \addplot {2*x+4};
  \end{axis}
\end{tikzpicture}
\end{prob}

\begin{prob}
D\^{e} o conjunto imagem das seguintes fun\c{c}\~{o}es:
\begin{multicols}{3}
\begin{enumerate}[(a)]
\item    $f\left( x\right) =x^{2}-4$
\item    $f\left( x\right) =x^{2}-5x+6$
\item    $f\left( x\right) =x^{2}+x+1$
\end{enumerate}
\end{multicols}
\end{prob}

\begin{prob}
Dada a fun\c{c}\~{a}o $g:\{ -1;2;4;5\} \rightarrow \{-7;-4;-1;0;4;8;17\}$, definida pela regra $g(x)=x^{2}-8$, indique o dom\'{\i}nio, o contradom\'{\i}nio e a imagem dessa fun\c{c}\~{a}o.
\end{prob}

\begin{prob}
Dado que o gr\'{a}fico da fun\c{c}\~{a}o $f\left( x\right) =\dfrac{x^{2}-3}{x+a}$ passa pelo ponto $(3;3)$, calcule o valor de $a$
\end{prob}

\begin{prob}
Uma fun\c{c}\~{a}o $f$ de vari\'{a}vel real satisfaz a condi\c{c}\~{a}o $f\left( x + 1\right) = f\left( x\right) + f\left( 1\right) $, qualquer que seja o valor da vari\'{a}vel $x$. Sabendo que $f\left( 2\right) = 1$, determine o valor de:
\begin{multicols}{3}
\begin{enumerate}[(a)]
\item    $f\left( 1\right)$
\item    $f\left( 7\right)$
\item    $f\left( x\right)$
\end{enumerate}
\end{multicols}
\end{prob}

\begin{prob}
Um proj\'{e}til \'{e} lan\c{c}ado e tem a sua altura (em metros) dada pela fun\c{c}\~{a}o $h\left( t\right) =100t-5t^{2}$. A partir desta calcule:
\begin{multicols}{2}
\begin{enumerate}[(a)]
\item    O tempo em que o proj\'{e}til permanece no ar
\item    A altura m\'{a}xima atingida pelo  proj\'{e}til
\end{enumerate}
\end{multicols}
\end{prob}

\begin{prob}
Uma fun\c{c}\~{a}o $f\left( x\right) =ax+b$ passa pelos pontos $\left( 3;5\right)$ e $\left( 5;9\right) $. Calcule $a$ e $b$ e desenhe o gr\'{a}fico da equa\c{c}\~{a}o.
\end{prob}

\begin{prob}
Daniel pretende contratar um plano de telefonia fixa. A empresa A cobra R\$25,00 de mensalidade fixa mais R\$0,10 por minuto de liga\c{c}\~{a}o. A empresa B cobra R\$35,00 de mensalidade fixa e mais R\$0,05 por minuto de liga\c{c}\~{a}o. Sendo assim, calcule o n\'{u}mero m\'{\i}nimo de minutos que Daniel deve usar para que o plano da empresa B seja mais vantajoso do que o da empresa A. Represente graficamente o resultado.
\end{prob}

\begin{res}
\begin{multicols}{2}

\begin{enumerate}[(1)]

\item\begin{enumerate}[(a)]
\item $55$
\item $7 - 4 a + 2 a^{2} - 4 b + 4 a b + 2 b^{2}$
\item $2x^{2}+4x+7$
\end{enumerate}

\item\begin{enumerate}[(a)]
\item $x^{2}+5x+6$
\item $x^{2}-x+3$
\end{enumerate}

\item $\dfrac{f\left( x+h\right) -f\left( x\right) }{h}=2x+h$

\item $f\left( 20\right)=44$

\item\begin{enumerate}[(a)]
\item $Im=[-4;+\infty[$
\item $Im=[-\dfrac{1}{4};+\infty[$
\item $Im=[\dfrac{3}{4};+\infty[$
\end{enumerate}
\columnbreak
\item \begin{itemize}
  \item Dom\'{\i}nio = $\{ -1;2;4;5\}$
  \item Contradom\'{\i}nio = $\{-7;-4;-1;0;4;8;17\}$
  \item Imagem = $\{-7;-4;8;17\}$
\end{itemize}

\item $a=-1$

\item\begin{enumerate}[(a)]
\item $f\left( 1\right)=\dfrac{1}{2}$
\item  $f\left( 7\right)=\dfrac{7}{2}$
\item  $f\left( x\right)=\dfrac{x}{2}$
\end{enumerate}

\item\begin{enumerate}[(a)]
\item $t=20s$
\item  $h_{max}=500m$
\end{enumerate}

\item $a=2$ e $b=-1$

\item $200$ minutos

\end{enumerate}
\end{multicols}
\end{res}
\end{document}
